\chapter{Introduction}
\section{Problem Background}
The background of this project lies in the requirement for an absolutely robust method of communication with submarines. Due to the nature of how submarines operate, it is considered optimal to be able for the submarine to receive messages without having to come to the surface, as this quickly reveals the position of the submarine. The current solution which has been in use since the early 20th century, this method uses radio transmissions in the Very Low Frequency band, commonly referred to as long wave. These signals have a range of hundreds of kilometres. Up until the 1960's the modulation method was a simple on-off keying method, otherwise known as Morse code. In the 1950's Collins radio developed Minimum Shift Keying, which is a phase based carrier modulation technique which has been widely adopted in communications. The  advantage of VLF radio is that it can penetrate through several metres of water meaning that a submarine can receive a transmission by towing an aerial underwater. It is also highly resistant to the fallout radiation from a nuclear explosion, unlike other high frequency communications.
\\\\
VLF communications are particularly prone to interference from the broadband electromagnetic radiation created by lightning events leading to symbol loss at the receiver. Due to the size of transmitters required to broadcast signals with such a low frequency, VLF communications are only one way, shore-to-ship. This means that conventional error correction techniques where the receiver can simply request a re-transmit is simply not possible, the current mitigation strategy for this means that the message is simply rebroadcast many times. However due to the narrow bandwidth of VLF communications the data rate is very low typically around 35 ASCII/UTF-8 characters per second each character having 8 bits. This means that the submarine may have to spend a long time close to the surface in order to receiver the message enough times in order to complete it. Here lies the requirement to further investigate these signals in order to develop a tool that can increase the robustness of symbol recovery and estimate where in the signal symbol loss may have occurred.

\section{Introduction to the Theory}
The basis of this problem lies in the recovery of a square-wave/Piecewise signal from a noise corrupted signal. Minimum Shift keying which is the method of frequency modulation used for sending radio messages using VLF
