\renewcommand{\abstractname}{Summary}
\begin{abstract}
    Radio communications in the Very Low Frequency band are used for communications with submarines. These communications have a range of hundreds of kilometres and originate from several large transmitters at locations all around the globe. The method of communication is through a frequency modulation technique called Minimum Shift Keying which modulates the signal in phase as well as frequency. This modulation technique is spectrally efficient and makes it fairly robust to interference, an exception to this being lightning interference. Lightning produces impulsive high energy broadband electromagnetic interference which encompasses the 3-30kHz band used for these communications. This has the potential to cause jamming of the signal resulting in bit loss at the receiver. This is currently mitigated using error correction embedded in the message as part of the encryption process. This investigation aims to develop a method of non-data aided error correction scheme that may ultimately help reduce the complexity of these communication systems.
    \\\\
    The experimental aspect of this investigation used a combination of real world and simulated data in order to develop an understanding of signal characteristics and identify the manner in which they are affected by the interference from lightning. A symbol recovery tool was developed, using a differential demodulation technique, that provided good results, in line with state-of-the art methods. In the presence of impulsive noise bit loss still occurred. A strategy to mitigate bit loss was developed, by the development of local area statistics from the received signal. Areas of interference are isolated and linear interpolation applied to the phase to eliminate false peaks in the linear progression. 
    \\\\
    Implementing this method showed a mean bit error rate improvement of 33\%. Further analysis highlighted that in all instances of a disruption in the linear phase progression the bit loss was mitigated. However, in some instances continuous interference occurring around a bit transition caused these points to be falsely identified which led to errors in the bit recovery, compared with the recovery of the unmodified phase. This method, however had the overall effect of improving the bit error rate.
    
\end{abstract}