\chapter{Literature Review}

A variety of literature has been reviewed in order to provide the necessary base of knowledge in order to proceed with this project. This has been developed based around three broad areas of understanding required to develop the project.

\begin{enumerate}[i]
    \item \textbf{Atmospheric Interference} - The nature of lightning and relevant information related to modelling it. With particular interest in low frequency interference.
    \item \textbf{Signal Processing} - Literature covering the theory of complex modulation and demodulation and additionally techniques relating to recovering the bit sequence.
    \item \textbf{Practical Communications} - Understanding the real world equipment and practices. Due to the secretive nature of submarine operations some information will have to be inferred from this.
\end{enumerate}

\section{Atmospheric Interference}
The nature of atmospheric interference as stated by \cite{Rakov2013} is caused by lightning events that produce broadband electromagnetic radiation between 1Hz and 300MHz, within 200km of a lightning event. There are two main types of lightning event, Intra-Cloud and Cloud-Ground. These events and their respective sub-events generate electromagnetic radiation at different places within this band. Of particular interest is the activity around 10KHz as \cite{Fullekrug2013} as this is the return stroke of the lightning lasting ~20ms.
 In the late \nth{20} century NATO renewed their interest in VLF communication \cite{Posterijen1964} states the requirement of improved models of atmospheric interference in order to improve the performance of this communication especially in regions of the world where satellite communications are less reliable or non existent.
 Atmospheric electromagnetic radiation can be analysed in two discrete forms. Firstly this can be measured over a long scale in order to assess the seasonal and diurnal variation in background interference. Key studies relating to this are from 2003,2007 and 2010. \cite{berenger2002} proposes a computational method using finite differencing in order to analyse wave propagation, critically much faster than traditional full wave analysis. \cite{Fieve2007} discusses the results of a global survey into background radiation, highlighting the global diurnal and seasonal variation. Highlighting equatorial areas as particularly areas of high activity with the main exception being the north-east Atlantic. \cite{Reuveni2010} Does a similar study however data is recovered on a much smaller scale, only in the desert. The purpose of this study is looking at the spikiness of noise, outcome of this was that for the band 24Khz $\pm37.5\si{\hertz}$ had the lowest diurnal variation. Both studies fit Gaussian distributions to their data. \cite{Chrissan2000} looks at the noise as impulsive stating that below 100MHz the Gaussian model is appropriate, looking in particular into what could be seen as the 'instantaneous noise' occurring during a lightning event. He makes use of APD's in order to try and assess an appropriate statistical model to represent the problem. The conclusions are that the impulsive properties of lightning or non-uniform, and a synthetic-alpha-stable model is an appropriate two parameter model that fits the data.

\section{Signal Processing}
There are various methods of extracting the symbols from the baseband signal. They can be predominantly split into two forms, geometric and probabilistic. 
As a real application \cite{Adlard1999} performs a brief study that looks at the GMSK signals, but mainly looks at removing adjacent channel interference with the presence of impulsive noise. The conclusions are that impulsive noise is difficult to remove without adversely affecting the FRESH filtering technique.
\cite{Park2011} introduces differential detection techniques for which QPSK of which MSK is a variant, evaluating the bit error rate against signal to noise ratio.
\cite{Little2011} introduces several generalised methods for extracting piecewise constant signals.
\cite{Wiklundh2015} looks at adapting a log likelihood receiver as a way of improving the output of a differential demodulator for a LDPC decoder. \cite{Forney1973} gives an explanation and tutorial of the viterbi algorithm which is an implementation of a maximum likelihood technique described succinctly by \cite{weisstein}, which broadly covers most probabilistic techniques of symbol recovery. The viterbi algorithm makes an estimate of what the bit sequence is for a given signal. \cite{Shen2015} looks at the orthogonal matching pursuit algorithm as a way of recovering signals from noisy measurements. \cite{Yue2016} presents a differential demodulation strategy that's performance is irrelevant of whether the carrier phase is detected.\cite{Yang2016} investigates the effectiveness of this in an impulsive noise environment, simulating the noise as a synthetic-alpha-stable distribution, and the effect on BER for coherent and non-coherent detection (where the carrier phase is known and when it is not). 
\cite{Liu2018} utilises the lightning waveform bank to calculate the complex waveform using differential techniques to obtain the instantaneous frequency and phase. Comparing data from receivers globally it sets the precedent for a time differencing method based off frequency deviation in order to calculate lightning distances. \cite{Koh2018} offers supporting work, looking in depth at the behaviour of the received man-made signal from one specific VLF transmitter, in order to generate further understanding of the lower ionosphere propogation of lightning sferics.

\section{Practical Communications}
\cite{appendixA} is a specification document relating to the equipment carried on board US nuclear submarines, specifically relating to communications. As a NATO member it is assumed that equipment and communication techniques will be broadly similar accross other NATO members such as the UK and France (the only other two that operate nuclear submarines). From this document the communication parameters can be inferred, most importantly the operating bandwidth of 200-800 bits/sec, this could be four parallel channels of 200Hz. The principles of MSK are highlighted clearly by \cite{Pasupathy1979}, in this he also discusses FSK and QPSK of which MSK is an adapted form to provide higher spectral efficiency and similar BER to BPSK including high bit rate per Hz of bandwidth. 
The bandwidth of the MSK channel determines the switching frequency, it occurs when the two side frequencies coincide $fc\pm\frac{BW}{2}$.

\section{Project Context}
The context of the work to be undertaken in the context of the literature is that there is clearly a requirement to investigate further the nature of the electromagnetic radiation generated by lightning events, as most of the studies are long term studies that are trying to use a gaussian model for seasonal behaviour. The work of most interest is that of \cite{Chrissan2000} regarding the use of APD's from recorded data, which are much more representative of impulsive noise supported by \cite{Yang2016} using the S-$\alpha$-S model for impulsive noise. The limitation of Yang's work in particular is that it uses a form of Maximum Likelihood detection requiring the implementation of some complex mathematics, however the results are useful for a benchmark comparison. 
\\\\
Regarding signal recovery most of the methods although effective mainly stem from Maximum Likelihood which as previously mentioned is considered to be limited by it's complexity. However there are methods relating to using the orthogonal properties of the extracted signal, for this investigation this is the most practical route to follow as the original message will be in the form of a piecewise/square signal representing a binary sequence.
\\\\
In conclusion the literature provides a very good overview of the problem in particular regarding the theory of methodology of using techniques such as differential demodulation, and the nature of the interaction of the lightning sferics with man-made signals.